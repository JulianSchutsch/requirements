\documentclass[12pt]{article}
\usepackage{parskip}
\usepackage{graphicx}
\begin{document}
\section{Introduction}
The requirements tool is developed following the unix philosophy of doing one thing and one thing well.

The underlying data, hierarchical texts and annotations, are too complicated to be processed using a single text file. To simplify cooperation of programs, we introduce two ways,
\begin{itemize}
\item A library simplifying access to the requirement folders
\item A console tool to allow indirect usage of the library through script languages
\end{itemize}

This document describes the architecture and the library and a few usage scenarios.

\section{Layer architecture}
\begin{figure}[!t]
\includegraphics[width=1.0\textwidth]{layer_architecture.pdf}
\caption{Layer architecture of the requirement system}
\label{fig:layerarchitecture}
\end{figure}

The layer architecture shown in fig. \ref{fig:layerarchitecture} insulates the application using library from lower layers as much as possible.
Under ideal circumstances the application can interact with the batch system
using only commands. In some cases this is not enough, hence the node collection
and the results from the annotation system are available to the application as well. These results are clones with ownership transfered to the application.

The application creates instances of commands and sends them to the batch system for execution. usually the batch system runs within its own thread, therefore this operation is non blocking.

The batch system calls the annotation system's parser after all commands from its queue have been processed. Once this step is finished, the application receives a callback containing a status.

The application must refrain from using any lower layers directly. Only commands
must access the lower layers as they can do so in a safe manner.

\section{Storage/Blobs}
Storage (::requirements::IStorage) abstracts the details of how nodes and their relationships are stored. The IStorage object should only be managed by the Status object instanciated by the batch system.

A programmer of a command therefore uses the storage system like this,
\begin{verbatim}
void MyCommand::execute(Status& status) {
  auto storage = status.openStorage();
  // ... Code working on the model
  //     accessible through storage
  // ...
}
\end{verbatim}

The storage interface offers methods to work with blobs and to retrieve a reference to the node collection.

Once the command finishes, the storage will take care of saving all changes.

It is not possible to execute a command and revert its effect afterwards.
It is recommended to use a version control system for the storage folder.

\section{Node Collection}
The node collection contains a tree of nodes, each associated with content (simple string). The simplicity of this model is on purpose since the complicated details of reliable processes are added through the annotation system.

As an example, one could implement a simple command, switching the contents of two nodes,
\begin{verbatim}
void SwitchContent::execute(Status& status) {
  auto storage = status.openStorage();
  auto& collection = storage->getNodeCollection();
  auto node1 = collection.getNodeById(id1);
  auto node2 = collection.getNodeById(id2);
  auto temp = node1.getContent();
  node1.updateContent(node2.getContent());
  node2.updateContent(temp);
}
\end{verbatim}

The \texttt{id1} and \texttt{id2} are supposed to be members of SwitchContent, likely passed to the constructor from the application. Selection helpers are available to handle a larger set of common cases.

\section{Annotation System}
The annotation system is a collection of annotations for different purposes.
Annotations are the result of parsing the content of the node collection.

There are annotations for several purposes, e.g.
\begin{itemize}
 \item Associate nodes with shortcuts
 \item Associate nodes with error messages (parser errors)
 \item Associate nodes with status, e.g. if a requirement is covered by
  acceptance criteria
\end{itemize}

Important to know is, that annotations are not stored, they are derived from the
node collection and external sources (e.g. test results) any time its required.

They are essential in the creation of reports and extending the gui's views on the raw content.

\end{document}